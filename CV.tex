\documentclass{resume} % Use the custom resume.cls style

\newcommand{\AuthorName}{Darien J. Morrow}
\newcommand{\AuthorLastName}{Morrow } % for footer. 
\newcommand{\resumetype}{CV. }


\name{\AuthorName}
\address{Postdoctoral fellow at Argonne National Laboratory's Center for Nanoscale Materials}
%\address{\textrm{1101 University Ave, Madison, WI 53706}}
\address{ORCID: \href{http://orcid.org/0000-0002-8922-8049}{0000-0002-8922-8049}}
%\address{\href{http://darien.fyi}{\texttt{darien.fyi}}}
\address{\href{mailto:darienmorrow@gmail.com}{\texttt{darienmorrow@gmail.com}} $\vert$ \href{mailto:dmorrow@anl.gov}{\texttt{dmorrow@anl.gov}}}




\begin{document}

\begin{rSection}{Education}

\begin{rSubsection}{University of Wisconsin--Madison}{2015 - 2020}{PhD: Physical Chemistry. \emph{GPA: 4.0/4.0}}{Madison, WI}
\item Dissertation title: Development of multidimensional spectroscopies to investigate transition metal dichalcogenide and lead halide perovskite semiconductors
\item Adviser: John C. Wright.
\end{rSubsection}

\begin{rSubsectionlistless}{Missouri Western State University}{2011-2015}{BS (Honors): Chemistry; Minors: Mathematics \& Physics. \emph{GPA: 4.0/4.0}}{Saint Joseph, MO} 
\end{rSubsectionlistless}



\end{rSection}


\begin{rSection}{Research \& Work Experience}
	
\begin{rSubsection}{Center for Nanoscale Materials at Argonne National Laboratory}{2020 - Present}{Postdoctoral Appointee}{Lemont, IL}
	\item Supervised by Xuedan Ma
	\item Researching excitonic and electronic properties of carbon nanotubes and  transition metal dichalcogenides
	\item Developing cryogenic magneto-optical microscopy techniques	
\end{rSubsection}

\begin{rSubsection}{John C. Wright Research Group}{2015 - 2020}{Graduate Assistant}{Madison, WI}
\item Pioneered spectrally resolved harmonic generation as a probe of semiconductor excited state dynamics
\item Developed a suite of ultrafast techniques to explore excited state dynamics of thin film semiconductors relevant to photovoltaics (lead halide perovskites and transition metal dichalcogenides)%---these techniques include transient transmittance, transient reflectance, and transient grating spectroscopies 
%\item Developing spectroscopic probes which are exquisitely sensitive to coherent energy/charge transfer between layers of a heterostructure
%\item Studying non-linear mixing in non-resonant media
\item Developed open-source software packages for the  collection, processing, and modeling of multidimensional spectra (see \href{http://github.com/wright-group}{\texttt{github.com/wright-group}})
%\item Practicing open science; for instance, see our work on GitHub: \href{http://github.com/wright-group}{\texttt{github.com/wright-group}} \& \href{https://github.com/darienmorrow}{\texttt{github.com/darienmorrow}} and the Open Science Framework: \href{http://osf.io/x9743/}{\texttt{osf.io/x9743}}
\item Responsible for maintenance and furtherance of custom ultrafast laser systems including construction of new optomechanical \& electronic hardware, training new users, and troubleshooting hardware \& software

\end{rSubsection}

\begin{rSubsection}{Christopher G. Elles Research Group}{2014}{REU Fellow}{Lawrence, KS}
	\item Investigated the excited state dynamics of substituted thiophene photo-rearrangement reactions
	\item Developed and implemented reaction quantum yield measurement technique
	\item Used ultrafast transient absorption spectroscopy to probe singlet and triplet excited state manifolds
\end{rSubsection}

\begin{rSubsection}{Michael W. Ducey Research Group}{2011 - 2012}{Undergraduate Assistant}{Saint Joseph, MO}
	\item Investigated the solvatochromism of room temperature ionic liquids (RTILs) in common solvents
	\item Demonstrated that solvents can induce order in the alkyl side chains of methylimidazolium RTILs
\end{rSubsection}

\begin{rSubsectionlistless}{Morrow Contracting and Construction LLC}{2011 - 2015}{Skilled Laborer}{Saint Joseph, MO}
\end{rSubsectionlistless}

\end{rSection}

\pagebreak

\begin{rSection}{Publications}
	
\begin{etaremune}

	
\item[] \textit{In preparation:} \textbf{Morrow, D. J.}; Kohler, D. D.; Zhao, Y.; Scheeler, J. M.; Jin, S.; Wright, J. C. Ultrafast, multidimensional pump-probe spectroscopy of atomically thin WS\textsubscript{2}-MoS\textsubscript{2} lateral heterostructures
	
%\item[]  \textit{In preparation:} Sunden, K. F.; Thompson, B. J.;  \textbf{Morrow, D. J.}; Kain, S.; ; Kohler, D. D.;  Wright, J. C.	Automated Optical Parametric Amplifier tuning: a case study in enabling multidimensional spectroscopy.

	
\item[] \textit{Submitted:} \textbf{Morrow, D. J.}; Kohler, D. D.; Zhao, Y.; Scheeler, J. M.; Jin, S.; Wright, J. C.	Quantum interference between the optical Stark effect and resonant harmonic generation in WS\textsubscript{2}.\\
$\cdot$ Preprint: \href{http://arxiv.org/abs/2006.01183}{\texttt{arXiv:2006.01183}}. \\
$\cdot$ Data and code repository: \href{https://osf.io/sntpc/}{\texttt{DOI 10.17605/OSF.IO/sntpc}}.




\item[] \textit{Submitted:} Pan, D.; Fu, Y.; Luo, Z.; Zhao, Y.; \textbf{Morrow, D. J.}; Roy, C.; Liu, B.; Chen, S.; Wright, J. C.; Pan, A.; Jin, S. Deterministic fabrication of arbitrary 2D Ruddlesden-Popper halide perovskite heterostructures with emergent interlayer properties

\item \textbf{Morrow, D. J.}; Hautzinger, M. P.; Lafayette, D. P.; Scheeler, J. M.; Dang, L.; Leng, M.; Kohler, D. D.; Wheaton, A. M.; Fu, Y.; Guzei, I. A.; Tang, J.; Jin, S.; Wright, J. C. Disentangling Second Harmonic Generation from Multiphoton Photoluminescence in Halide Perovskites using Multidimensional Harmonic Generation. \emph{Journal of Physical Chemistry Letters}. \href{https://pubs.acs.org/doi/10.1021/acs.jpclett.0c01720}{\texttt{DOI:10.1021/acs.jpclett.0c01720}}. \textbf{2020} \\
$\cdot$ Preprint: \href{https://dx.doi.org/10.26434/chemrxiv.12055440}{\texttt{DOI: 10.26434/ChemRxiv.12055440}}. \\
$\cdot$ Data and code repository: \href{https://osf.io/jn24u/}{\texttt{DOI 10.17605/OSF.IO/jn24u}}.

\item Hautzinger, M. P.; Pan, D.; Piggs, A. K.; Fu, Y.; \textbf{Morrow, D. J.}; Leng, M.; Kuo, M.; Spitha, N., Lafayette, D. P.; Kohler, D. D.; Wright, J. C.; Jin, S. Band Edge Tuning of 2D Ruddlesden-Popper Perovskites by A Cation Size Revealed through Nanoplates. \emph{ACS Energy Letters}. \href{https://pubs.acs.org/doi/10.1021/acsenergylett.0c00450}{\texttt{DOI:10.1021/acsenergylett.0c00450}}. \textbf{2020} \\
$\cdot$ Code repository: \href{https://osf.io/m9dnw/}{\texttt{DOI 10.17605/OSF.IO/m9dnw}}.

\item  \textbf{Morrow, D. J.}; Kohler, D. D.; Zhao, Y.; Jin, S.; Wright, J. C. Triple sum frequency pump-probe spectroscopy of transition metal dichalcogenides. \emph{Physical Review B}. \href{https://journals.aps.org/pra/abstract/10.1103/PhysRevB.100.235303}{\texttt{DOI: 10.1103/PhysRevB.100.235303}}. \textbf{2019}.\\
$\cdot$ Preprint: \href{http://arxiv.org/abs/1909.06445}{\texttt{arXiv:1909.06445}}. \\
$\cdot$ Data and code repository: \href{https://osf.io/UMSXC/}{\texttt{DOI 10.17605/OSF.IO/UMSXC}}.

\item Thompson, B. J.; Sunden, K. F.; \textbf{Morrow, D. J.}; Kohler, D. D.; Wright, J.C. 
WrightTools: a Python package for multidimensional spectroscopy \emph{The Journal of Open Source Software}. 
\href{http://doi.org/10.21105/joss.01141}{\texttt{DOI: 10.21105/joss.01141}}. \textbf{2019}.
	
\item \textbf{Morrow, D. J.}; Kohler, D. D.; Czech, K. J.; Wright, J. C. 
Communication: Multidimensional Triple Sum-Frequency Spectroscopy of MoS\textsubscript{2} and Comparisons with Absorption and Second Harmonic Generation Spectroscopies. \emph{Journal of Chemical Physics}. \href{http://doi.org/10.1063/1.5047802}{\texttt{DOI: 10.1063/1.5047802}}. \textbf{2018}.\\
$\cdot$ Preprint: \href{http://arxiv.org/abs/1805.06985}{\texttt{arXiv:1805.06985}}. \\
$\cdot$ Data and code repository: \href{https://osf.io/2wf6g/}{\texttt{DOI 10.17605/OSF.IO/2WF6G}}.
	
\item \textbf{Morrow, D. J.}; Kohler, D. D.; Wright, J. C. Group and phase velocity mismatch fringes in triple sum-frequency spectroscopy. \emph{Physical Review A}. \href{https://journals.aps.org/pra/abstract/10.1103/PhysRevA.96.063835}{\texttt{DOI: 10.1103/PhysRevA.96.063835}}. \textbf{2017}.\\
$\cdot$ Preprint: \href{http://arxiv.org/abs/1709.10476}{\texttt{arXiv:1709.10476}}. \\
$\cdot$ Data and code repository: \href{https://osf.io/emgta/}{\texttt{DOI 10.17605/OSF.IO/EMGTA}}.

\item Fu, Y.; Rea, M. T.; Chen, J.; \textbf{Morrow, D. J.}; Hautzinger, M. P.; Zhao, Y.; Manger, L. H.; Wright, J. C.; Goldsmith, R. H.; Jin, S. Selective Stabilization and Photophysical Properties of
Metastable Perovskite Polymorphs of CsPbI\textsubscript{3} in Thin Films. \emph{Chem. Mater.} \href{http://pubs.acs.org/doi/10.1021/acs.chemmater.7b02948}{\texttt{DOI: 10.1021/acs.chemmater.7b02948}}. \textbf{2017}. 
 
\item Chen, J.; \textbf{Morrow, D. J.}; Fu, Y.; Zheng, W.; Zhao, Y.; Dang, L.; Stolt, M. J.; Kohler, D. D.; Wang, X.; Czech, K. J.; Hautzinger, M. P.; Shen, S.; Guo, L.; Pan, A.; Wright, J. C.; Jin, S. Single-Crystal Thin Films of Cesium Lead Bromide Perovskite Epitaxially Grown on Metal Oxide Perovskite (SrTiO\textsubscript{3}). \emph{J. Am. Chem. Soc.} \href{http://pubs.acs.org/doi/10.1021/jacs.7b07506}{\texttt{DOI: 10.1021/jacs.7b07506}}. \textbf{2017}. \\
$\cdot$ Data and code repository: \href{https://osf.io/v5kzn/}{\texttt{DOI 10.17605/OSF.IO/V5KZN}}.
	


\pagebreak
\end{etaremune}


\end{rSection}

\begin{rSection}{Patents}
	\begin{etaremune}
		\item[] \textit{U.S. Patent Pending, filed 2019-06-20} \textbf{Morrow, D. J.}; Kohler, D. D.; Wright, J. C. Ultrafast, multiphoton-pump, multiphoton-probe spectroscopy. 		
	\end{etaremune}	
\end{rSection}


\begin{rSection}{Posters \& Presentations}
	
	
\begin{etaremune}
	\item Poster. \textbf{Darien J. Morrow}, Daniel D. Kohler, John C. Wright. Development of sum-frequency and transient sum-frequency spectroscopies to study transition metal dichalcogenide nanostructures. ACS National Meeting, Philadelphia, PA, March 2020. (meeting was canceled due to COVID-19) 
	
	\item Poster. \textbf{Darien J. Morrow}, Daniel D. Kohler, John C. Wright. Multi-photon pump, multi-photon probe spectroscopies and their application to MX\textsubscript{2} nanostructures. CMDS 2018, Seoul, South Korea. June 2018.

	\item Poster. \textbf{Darien J. Morrow}, Jenna M. Wasylenko, Christopher G. Elles. Kinetics and
	Dynamics of the Photorearrangement Reactions of Aryl-Substituted Thiophenes. ACS National Meeting, Denver, CO. March 2015. 
	
	\item Poster. Michael W. Ducey, \textbf{Darien J. Morrow}, Bethany Thornton, Varun Lahoti. Conformational behavior and applications of mixed room temperature ionic liquid solvent systems examined with a panel of solvatochromic probes. ACS Midwest Regional Meeting, Columbia, MO. November 2014.
	
	\item Poster. \textbf{Darien J. Morrow}, Jenna M. Wasylenko, Christopher G. Elles. Kinetics and Dynamics of the Photorearrangements of Conjugated Thiophenes. Council on Undergraduate Research, Research Experiences for Undergraduates Symposium, Arlington, VA. October 2014.
	
	%\item Poster. \textbf{Darien J. Morrow}, Jenna M. Wasylenko, Christopher G. Elles. Kinetics and Dynamics of the Photorearrangements of Conjugated Thiophenes. The University of Kansas, REU Poster Session, Lawrence, KS. July 2014. 
	
	%\item Poster. Alexander K. Moore, \textbf{Darien J. Morrow}. How to Play Rock Paper Scissors Lizard Spock Like a Rational Person. Missouri Western State University, Multidisciplinary Research Symposium, St. Joseph, MO. May 2014.
	
	\item Poster. Melanie Edlin, David J. Freeman, Nathan Harms, Xu Ho, Torin McKinley, Alexander Moore, \textbf{Darien J. Morrow}, Christopher Phillips, Jeffrey N. Woodford, Determination of Dimerization Constant of N-(isoquinolin-3-yl)Benzamide and N-(isoquinolin-2-yl)Benzamide. ACS Midwest Regional Meeting, Springfield, MO. October 2013.
	
	%\item Poster. \textbf{Darien J. Morrow}, Michael W. Ducey, Solvatochromic Properties of Ionic Liquid: Solvent and Polymer Systems Examined with PRODAN. Missouri Western State University, Multidisciplinary Research Symposium, St. Joseph, MO. May 2012.
	
\end{etaremune}
\end{rSection}

%\pagebreak


\begin{rSection}{Teaching Experience}
	
\begin{rSubsectionlistless}{Physical Chemistry: Thermodynamics}{Fall 2016}{Teaching Assistant for Prof. Gilbert M. Nathanson }{Madison, WI}
\end{rSubsectionlistless}

\begin{rSubsectionlistless}{General Chemistry}{Fall 2015 - spring 2016}{Teaching Assistant for Prof. Ive Herman and Dr. Paul Hooker}{Madison, WI}
\end{rSubsectionlistless}

\begin{rSubsectionlistless}{Organic Chemistry II}{Fall 2013}{Teaching Assistant for Prof. Steven P. Lorimor}{Saint Joseph, MO}
\end{rSubsectionlistless}

\end{rSection}


\begin{rSection}{Fellowships \& Scholarships}
	
\begin{itemize}[leftmargin = 0 pt]
	\item Link Foundation Energy Fellowship. July 2018 - June 2020. \\
	Two year full stipend for \emph{Investigation of Coherent Charge Transfer in Transition Metal Dichalcogenide Heterostructures with Multiresonant Coherent Multidimensional Spectroscopy}.
	\item Pei Wang Fellowship. Fall 2015 - spring 2016.
	\item Golden Griffon Honors scholarship. Fall 2011 - spring 2015.
	\item NSF funded Midwest Apex Project scholarship. Fall 2011 - spring 2015.
	\item Missouri Bright Flight scholarship. Fall 2011 - spring 2015.
	%\item J. B. Bruce scholarship.
	%\item Marion Mitchell scholarship.
	%\item Hillyard Chemistry scholarship.
	%\item Western Excellence Chemistry scholarship.
\end{itemize}
	
\end{rSection}

\pagebreak
\begin{rSection}{Awards \& Honors}

\begin{itemize}[leftmargin = 0 pt]
	\item UW--Madison Department of Chemistry, Richard and Joan Hartl Award for Research Excellence in Physical Chemistry. 2020.
	\item UW--Madison Department of Chemistry, Roger Carlson Memorial Award for Excellence in Analytical Chemistry. 2018.
	\item NSF Graduate Research Fellowship Program, Honorable mention. 2017.
	\item MWSU Department of Chemistry, Edgar C. Little Outstanding Student Award. 2015.
	\item ACS Division of Analytical Chemistry, Undergraduate Award in Analytical Chemistry. 2015. 
	\item ACS Division of Inorganic Chemistry, Undergraduate Award in Inorganic Chemistry. 2013.
	\item MWSU President’s Honor’s List. Fall 2011 - spring 2015.
\end{itemize}

\end{rSection}


\begin{rSection}{Software Skills}
	\begin{itemize}[leftmargin = 0 pt]
		\item Python and the scientific Python software stack (numpy, matplotlib, scipy, h5py)
		\item Working knowledge: Arduino, Git, Latex, Autodesk Inventor
		\item Active contributor/maintainer of open source projects:
		\begin{itemize}
			\item \href{http://wright.tools}{\texttt{WrightTools}} (library): loading, processing, and plotting of multidimensional spectroscopy data 
			\item \href{http://github.com/wright-group/PyCMDS}{\texttt{PyCMDS}} (application): orchestrating many hardware into multidimensional spectrometers  
			\item \href{http://github.com/wright-group/attune}{\texttt{attune}} (library): tuning/calibrating multidimensional spectrometers
		\end{itemize}

	\end{itemize}	
\end{rSection}

%\newpage

\begin{rSection}{Service Activities \& Community Involvement}

\begin{itemize}[leftmargin = 0 pt]
	\item Hosted ``Detector Building'' competition at the Science Olympiad Regional Tournament. Madison, Wisconsin. Winter 2020
	\item Organized weekly seminar for physical chemistry graduate students to present their research to fellow graduate students. 2018-2019
	\item Served as a moderator for the annual Wisconsin Middle School Science Bowl (sponsored by the DOE). 2017-present.
	\item Wisconsin Institute for Discovery volunteer. 2017-present. 
	\item Taught/supervised electronics for a week to high schoolers in the PEOPLE program. Summer 2017.
	\item Served on panel to talk to REU students about experiences applying to and surviving graduate school. Summer 2017.
	\item Talked and demonstrated to Institute of Chemical Education summer camp attendees about my research, renewable energy, and how solar cells work. Summer 2017.  
	\item Served as vice-president (2014-2015) and member of Missouri Western State University's ACS affiliated Chemistry club. 2011-2015.
	\item Aided in the organization and implementation of Super Science Saturday and Chemathon at Missouri Western State University. 2011-2015.	
\end{itemize}

\end{rSection}

\pagebreak

\begin{rSection}{References}
	\begin{itemize}[leftmargin = 0 pt]
		\item Prof. John C. Wright (Doctoral advisor) | \href{mailto:wright@chem.wisc.edu}{\texttt{wright@chem.wisc.edu}} | 608-262-0351
		\begin{itemize}
			\item[] Department of Chemistry
			\item[] University of Wisconsin--Madison
			\item[] 1101 University Ave Rm 3209
			\item[] Madison, WI 53706
		\end{itemize}
		\item Dr. Xuedan Ma (Postdoctoral advisor) | \href{mailto: xuedan.ma@anl.gov}{\texttt{ xuedan.ma@anl.gov}} | 630-252-3716
		\begin{itemize}
			\item[] Center for Nanoscale Materials
			\item[] Argonne National Laboratory
			\item[] 9700 S. Cass Avenue. Building 440 Rm A242
			\item[] Lemont, IL 60439
		\end{itemize}	
		\item Prof. Martin T. Zanni | \href{mailto:zanni@chem.wisc.edu}{\texttt{zanni@chem.wisc.edu}} | 608-262-4783
		\begin{itemize}
			\item[] Department of Chemistry
			\item[] University of Wisconsin--Madison
			\item[] 1101 University Ave Rm 8305L
			\item[] Madison, WI 53706
		\end{itemize}
		
		%\item Prof. Christopher G. Elles | \href{mailto:elles@ku.edu}{\texttt{elles@ku.edu}} | 785-864-1922
		%\begin{itemize}
		%	\item[] Department of Chemistry
		%	\item[] The University of Kansas
		%	\item[] Malott Hall Room B031
		%	\item[] 1251 Wescoe Hall Dr.
		%	\item[] Lawrence, KS 66045 
		%\end{itemize}		
		\item Prof. Gilbert M. Nathanson (Teaching reference) | \href{mailto:nathanson@chem.wisc.edu}{\texttt{nathanson@chem.wisc.edu}} | 608-262-8098
		\begin{itemize}
			\item[] Department of Chemistry
			\item[] University of Wisconsin--Madison
			\item[] 1101 University Ave Rm 7321A
			\item[] Madison, WI 53706
		\end{itemize}
		%\item Prof. Deniz D. Yavuz | 	\href{mailto:yavuz@wisc.edu}{\texttt{yavuz@wisc.edu}} | 608-263-9399
		%\begin{itemize}
		%	\item[] Department of Physics
		%	\item[] University of Wisconsin--Madison
		%	\item[] 1150 University Ave Rm 5320
		%	\item[] Madison, WI 53706
		%\end{itemize}
	\end{itemize}
\end{rSection}


\end{document}
