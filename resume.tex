\documentclass{resume} % Use the custom resume.cls style

\newcommand{\AuthorName}{Darien J. Morrow}
\newcommand{\AuthorLastName}{Morrow } % for footer. 
\newcommand{\resumetype}{Resume. }

\name{\AuthorName}
\address{Postdoctoral Appointee at Argonne National Laboratory's Center for Nanoscale Materials}


\address{\href{mailto:darienmorrow@gmail.com}{\texttt{darienmorrow@gmail.com}} $\vert$ \href{mailto:dmorrow@anl.gov}{\texttt{dmorrow@anl.gov}} $\vert$
	1-816-752-4270} 




\begin{document}

\begin{rSection}{Education}

\begin{rSubsection}{University of Wisconsin--Madison}{2015 - 2020}{PhD: Physical Chemistry. \emph{Advised by John C. Wright.} \emph{GPA: 4.0/4.0} }{Madison, WI}
\item Dissertation title: Development of multidimensional spectroscopies to investigate transition metal dichalcogenide and lead halide perovskite semiconductors
\end{rSubsection}

\begin{rSubsectionlistless}{Missouri Western State University}{2011-2015}{BS (Honors): Chemistry; Minors: Mathematics \& Physics. \emph{GPA: 4.0/4.0}}{Saint Joseph, MO} 
\end{rSubsectionlistless}



\end{rSection}


\begin{rSection}{Research \& Work Experience}

\begin{rSubsection}{Center for Nanoscale Materials at Argonne National Laboratory}{2020 - Present}{Postdoctoral Appointee supervised by Xuedan Ma}{Lemont, IL}
	\item Researching excitonic, polaronic, and electronic properties of doped low-dimensional materials as photonic sources for Quantum Information Science
	\item Developing cryogenic magneto-optical microscopy techniques 	
\end{rSubsection}

\begin{rSubsection}{John C. Wright Research Group}{2015 - 2020}{Graduate Assistant}{Madison, WI}
	\item Pioneered spectrally resolved harmonic generation as a probe of semiconductor excited state dynamics
	\item Developed a suite of ultrafast techniques to explore excited state dynamics of thin film semiconductors relevant to photovoltaics (lead halide perovskites and transition metal dichalcogenides)%---these techniques include transient transmittance, transient reflectance, and transient grating spectroscopies 
	%\item Developing spectroscopic probes which are exquisitely sensitive to coherent energy/charge transfer between layers of a heterostructure
	%\item Studying non-linear mixing in non-resonant media
	%\item Developed open-source software packages for collection, processing, and modeling of multidimensional data 
	%\item Practicing open science; for instance, see our work on GitHub: \href{http://github.com/wright-group}{\texttt{github.com/wright-group}} \& \href{https://github.com/darienmorrow}{\texttt{github.com/darienmorrow}} and the Open Science Framework: \href{http://osf.io/x9743/}{\texttt{osf.io/x9743}}
	\item Responsible for maintenance and furtherance of custom ultrafast laser systems including construction of new optomechanical \& electronic hardware, training new users, and troubleshooting hardware \& software
\end{rSubsection}

%\begin{rSubsection}{Christopher G. Elles Research Group}{2014}{REU Fellow}{Lawrence, KS}
%	\item Investigated the excited state dynamics of substituted thiophene photo-rearrangement reactions
%	\item Developed and implemented reaction quantum yield measurement technique
%	\item Used ultrafast transient absorption spectroscopy to probe singlet and triplet excited state manifolds
%\end{rSubsection}


%\begin{rSubsectionlistless}{Morrow Contracting and Construction LLC}{2011 - 2015}{Skilled Laborer}{Saint Joseph, MO}
%\end{rSubsectionlistless}

\end{rSection}

\begin{rSection}{Software Skills}
	\begin{itemize}[leftmargin = 0 pt]
		\item Python and the scientific Python software stack (numpy, matplotlib, scipy, h5py)
		\item Working knowledge: Arduino, Git, Latex, Autodesk Inventor
		\item Active contributor/maintainer of open source projects:
		\begin{itemize}
			\item \href{http://wright.tools}{\texttt{WrightTools}} (library): loading, processing, and plotting of multidimensional spectroscopy data 
			\item \href{http://github.com/wright-group/PyCMDS}{\texttt{PyCMDS}} (application): orchestrating many pieces of hardware into multidimensional spectrometers  
			\item \href{http://github.com/wright-group/attune}{\texttt{attune}} (library): tuning/calibrating multidimensional spectrometers
		\end{itemize}
		
	\end{itemize}	
\end{rSection}


\begin{rSection}{Patents}
	\begin{etaremune}
		\item[$\cdot$] \textit{U.S. Patent awarded, filed 2019-06-20} \textbf{Morrow, D. J.}; Kohler, D. D.; Wright, J. C. Ultrafast, multiphoton-pump, multiphoton-probe spectroscopy. 		
	\end{etaremune}	
\end{rSection}



\begin{rSection}{Publications}
	
\begin{etaremune}[topsep=0pt,itemsep=0pt,partopsep=0pt,parsep=0pt]

\item[-] \textit{In preparation:} \textbf{Morrow, D. J.}; Ma. X. Understanding interlayer exciton trapping in two-dimensional heterostructures with discrete, random-walk simulations

\item[-] \textit{In preparation:} \textbf{Morrow, D. J.}; et. al. Ultrafast, multidimensional pump-probe spectroscopy of atomically thin WS\textsubscript{2}-MoS\textsubscript{2} lateral heterostructures

\item[-] \textit{Submitted:} Pan, D.; Fu, Y.; Luo, Z.; Zhao, Y.; \textbf{Morrow, D. J.}; et. al. Deterministic fabrication of arbitrary 2D Ruddlesden-Popper halide perovskite heterostructures with emergent interlayer properties


\item \textbf{Morrow, D. J.}; Kohler, D. D.; Zhao, Y.; Scheeler, J. M.; Jin, S.; Wright, J. C.	Quantum interference between the optical Stark effect and resonant harmonic generation in WS\textsubscript{2}.  \emph{Physical Review B}. \href{https://journals.aps.org/pra/abstract/10.1103/PhysRevB.102.161401}{\texttt{DOI: 10.1103/PhysRevB.102.161401}}. \textbf{2020}.

\item \textbf{Morrow, D. J.}; et. al. Disentangling Second Harmonic Generation from Multiphoton Photoluminescence in Halide Perovskites using Multidimensional Harmonic Generation. \emph{Journal of Physical Chemistry Letters}. \href{https://pubs.acs.org/doi/10.1021/acs.jpclett.0c01720}{\texttt{DOI:10.1021/acs.jpclett.0c01720}}. \textbf{2020}

\item Hautzinger, M. P.; Pan, D.; Piggs, A. K.; Fu, Y.; \textbf{Morrow, D. J.}; et. al. Band Edge Tuning of 2D Ruddlesden-Popper Perovskites by A Cation Size Revealed through Nanoplates. \emph{ACS Energy Letters}. \href{https://pubs.acs.org/doi/10.1021/acsenergylett.0c00450}{\texttt{DOI:10.1021/acsenergylett.0c00450}}. \textbf{2020}. 

\item  \textbf{Morrow, D. J.}; et. al. Triple sum frequency pump-probe spectroscopy of transition metal dichalcogenides. \emph{Physical Review B}. \href{https://journals.aps.org/pra/abstract/10.1103/PhysRevB.100.235303}{\texttt{DOI: 10.1103/PhysRevB.100.235303}}. \textbf{2019}.

\item Thompson, B. J.; Sunden, K. F.; \textbf{Morrow, D. J.}; et. al. % Kohler, D. D.; Wright, J.C. 
WrightTools: a Python package for multidimensional spectroscopy \emph{The Journal of Open Source Software}. 
\href{http://doi.org/10.21105/joss.01141}{\texttt{DOI: 10.21105/joss.01141}}. \textbf{2019}.
	
\item \textbf{Morrow, D. J.}; et. al. %Kohler, D. D.; Czech, K. J.; Wright, J. C. 
Communication: Multidimensional Triple Sum-Frequency Spectroscopy of MoS\textsubscript{2} and Comparisons with Absorption and Second Harmonic Generation Spectroscopies. \emph{Journal of Chemical Physics}. \href{http://doi.org/10.1063/1.5047802}{\texttt{DOI: 10.1063/1.5047802}}. \textbf{2018}.
	
\item \textbf{Morrow, D. J.}; et. al. % Kohler, D. D.; Wright, J. C. 
Group and phase velocity mismatch fringes in triple sum-frequency spectroscopy. \emph{Physical Review A}. \href{https://journals.aps.org/pra/abstract/10.1103/PhysRevA.96.063835}{\texttt{DOI: 10.1103/PhysRevA.96.063835}}. \textbf{2017}.

\item Fu, Y.; Rea, M. T.; Chen, J.; \textbf{Morrow, D. J.}; et. al. %Hautzinger, M. P.; Zhao, Y.; Manger, L. H.; Wright, J. C.; Goldsmith, R. H.; Jin, S. Selective Stabilization and Photophysical Properties of
Metastable Perovskite Polymorphs of CsPbI\textsubscript{3} in Thin Films. \emph{Chem. Mater.} \href{http://pubs.acs.org/doi/10.1021/acs.chemmater.7b02948}{\texttt{DOI: 10.1021/acs.chemmater.7b02948}}. \textbf{2017}. 
 
\item Chen, J.; \textbf{Morrow, D. J.}; et. al. % Fu, Y.; Zheng, W.; Zhao, Y.; Dang, L.; Stolt, M. J.; Kohler, D. D.; Wang, X.; Czech, K. J.; Hautzinger, M. P.; Shen, S.; Guo, L.; Pan, A.; Wright, J. C.; Jin, S. 
Single-Crystal Thin Films of Cesium Lead Bromide Perovskite Epitaxially Grown on Metal Oxide Perovskite (SrTiO\textsubscript{3}). \emph{J. Am. Chem. Soc.} \href{http://pubs.acs.org/doi/10.1021/jacs.7b07506}{\texttt{DOI: 10.1021/jacs.7b07506}}. \textbf{2017}. 
	



\end{etaremune}

\end{rSection}



\iffalse
\begin{rSection}{Teaching Experience}
	
	\begin{rSubsectionlistless}{Physical Chemistry: Thermodynamics}{Fall 2016}{Teaching Assistant for Prof. Gilbert M. Nathanson }{Madison, WI}
	\end{rSubsectionlistless}
	
	\begin{rSubsectionlistless}{General Chemistry}{Fall 2015 - spring 2016}{Teaching Assistant for Prof. Ive Herman and Dr. Paul Hooker}{Madison, WI}
	\end{rSubsectionlistless}
	
	\begin{rSubsectionlistless}{Organic Chemistry II}{Fall 2013}{Teaching Assistant for Prof. Steven P. Lorimor}{Saint Joseph, MO}
	\end{rSubsectionlistless}
	
\end{rSection}
\fi



\begin{rSection}{Fellowships \& Scholarships}
	
\begin{itemize}[leftmargin = 0 pt]
	\item Link Foundation Energy Fellowship (full graduate school stipend). July 2018 - June 2020. 
	\item Pei Wang Fellowship. Fall 2015 - spring 2016.
	\item Golden Griffon Honors scholarship. Fall 2011 - spring 2015.
	\item NSF funded Midwest Apex Project scholarship. Fall 2011 - spring 2015.
	\item Missouri Bright Flight scholarship. Fall 2011 - spring 2015.
	%\item J. B. Bruce scholarship.
	%\item Marion Mitchell scholarship.
	%\item Hillyard Chemistry scholarship.
	%\item Western Excellence Chemistry scholarship.
\end{itemize}
	
\end{rSection}


\begin{rSection}{Awards \& Honors}

\begin{itemize}[leftmargin = 0 pt]
	\item Richard and Joan Hartl Award for Research Excellence in Physical Chemistry. 2020.
	\item Roger Carlson Memorial Award for Excellence in Analytical Chemistry. 2018.
	\item NSF Graduate Research Fellowship Program, Honorable mention. 2017.
	\item MWSU Department of Chemistry, Edgar C. Little Outstanding Student Award. 2015.
	\item ACS Division of Analytical Chemistry, Undergraduate Award in Analytical Chemistry. 2015. 
	\item ACS Division of Inorganic Chemistry, Undergraduate Award in Inorganic Chemistry. 2013.
	\item MWSU President’s Honor’s List. Fall 2011 - spring 2015.
\end{itemize}
\end{rSection}


%\newpage

\begin{rSection}{Service Activities \& Community Involvement}

\begin{itemize}[leftmargin = 0 pt]
	\item Hosted ``Detector Building'' competition at the Science Olympiad Regional Tournament. Winter 2020	
	\item Organized weekly seminar series physical chemistry graduate students. 2018-2019
	\item Served as a moderator for the annual Wisconsin Middle School Science Bowl. 2017-present.
	%\item Wisconsin Institute for Discovery volunteer. 2017-present. 
	\item Taught/supervised electronics for a week to high schoolers in the PEOPLE program. Summer 2017.
	%\item Served on panel to talk to REU students about experiences applying to and surviving graduate school. Summer 2017.
	\item Talked and demonstrated to Institute of Chemical Education summer camp attendees about my research, renewable energy, and how solar cells work. Summer 2017.  
	\item Vice-president (2014-2015) and member of MWSU's ACS affiliated Chemistry club. 2011-2015.
	\item Aided in the organization of Super Science Saturday and Chemathon at MWSU. 2011-2015.	
\end{itemize}
\end{rSection}

\end{document}
